\documentclass[12pt]{article}
\usepackage[utf8]{inputenc}
\usepackage{chemfig}
\usepackage[version=4]{mhchem}
\usepackage{amsfonts}
\usepackage{amsmath}
\usepackage{amssymb}
\usepackage{geometry}
\usepackage{mathabx}
\usepackage{relsize}
\usepackage{graphics}
\usepackage[colorlinks = true,
            linkcolor = blue,
            urlcolor  = blue,
            citecolor = blue,
            anchorcolor = blue]{hyperref}
%\usepackage{indentfirst}
\usepackage{tikz}
\usepackage{sidecap}
\usepackage{sidecap}
\geometry{
 a4paper,
 total={6.5in,0in},
 left= 15mm,
 top= 15mm,
 bottom=15mm,
 right = 15mm
 }
\title{Math for Energy}
\author{Marcos Perez}
\date{May 2022}

\begin{document}

\maketitle
\section{Daughter nucleus abundance in n-generation decay chain}
\subsection{Formula Proposition}
Where there 1 is one mole of the 0-th generation nucleus at $t=0$ there are $m_i$ moles of the $i-th$ generation nucleus at time $t$. We assume that the $i-th$ generation nucleus has a measurable e-folding time of $\lambda_i$. Additionally, $r_i$ is the decay rate of the $i-th$ generation nucleus in moles/time.
\begin{equation}
\begin{split}
m_0 = e^{\frac{t}{-\lambda_0}}\\
m_{i>0} = -\frac{1}{\lambda_0}\int_0^te^{t(\frac{1}{-\lambda_0}+\frac{1}{-\lambda_1})}a_{i-1}dt\\
r_i = -\frac{1}{\lambda_0}e^{\frac{t}{-\lambda_0}}a_i\\
a_i = \prod_{k=1}^{i}(1-e^{t\frac{1}{-\lambda_k}})
\end{split}
\end{equation}
\subsection{Formula Proof}
At a given time $t$ there are $m_i$ moles of the $i-th$ daughter nucleus. For $i>0$, there are $\nu_{i}$ daughter nuclei produced by the decay of the $i-1th$ generation in the decay chain, and this is the only source of any nuclei in the system. We assume there is no source of new 0-th generation nuclei. By the properties of radioactive decay
\begin{equation}\label{eqn:propertiesRad}
\begin{split}
m_i = \int_0^t P_id\nu_{i}\\
\nu_i = \int_0^t (1-P_{i-1})d\nu_{i-1}\\
\nu_0 = 0\quad \nu_1 = 1-P_0\\
\frac{d\nu_{i}}{dt} = (1-P_{i-1})\frac{d\nu_{i-1}}{dt}\forall i\in\mathbb{Z}, i>1\\
\end{split}
\end{equation}
Where the probability of a single i-th nucleus not decaying after a time $t$ is
\begin{equation}
\begin{split}
P_i = e^{-\frac{t}{\lambda_{i}}}\\
\end{split}
\end{equation}
Thus we make the substitutions to write the following recursive formula
\begin{equation}\label{eqn:recursiveForm}
\begin{split}
m_i = -\int_0^t P_i(1-P_{i-1})\frac{d\nu_{i-1}}{dt}dt\\
\end{split}
\end{equation}
Assuming there is no source of new 0-th generation nuclei
\begin{equation}
\begin{split}
m_1 = \int_0^te^{\frac{t}{-\lambda_1}}d\nu_0 = \int_0^te^{\frac{t}{-\lambda_1}}\frac{d}{dt}(1-P_0)dt = -\int_0^te^{\frac{t}{-\lambda_1}}\frac{dP_0}{dt}dt = \frac{1}{\lambda_0}\int_0^te^{\frac{t}{-\lambda_1}}e^{\frac{t}{-\lambda_0}}dt = \frac{\lambda_1}{\lambda_1+\lambda_0}(1-e^{-t(\frac{1}{\lambda_0}+\frac{1}{\lambda_1})}) \\
\end{split}
\end{equation}
For more insight on how to write the general case, we first compute using \ref{eqn:propertiesRad}
\begin{equation}\label{eqn:m2derivation}
\begin{split}
m_2 = \int_0^t P_2d\nu_{2} = \int_0^t e^{\frac{t}{-\lambda_2}}d\nu_2\\
d\nu_{2} = (1-P_{1})\frac{d\nu_{1}}{dt}dt\\
m_2 = \int_0^t e^{\frac{t}{-\lambda_2}}(1-P_{1})\frac{d\nu_{1}}{dt}dt = -\int_0^t e^{\frac{t}{-\lambda_2}}(1-P_{1})\frac{dP_0}{dt}dtdt\\
m_2 = \frac{1}{\lambda_0}\int_0^t e^{\frac{t}{-\lambda_2}}(1-e^{\frac{t}{-\lambda_1}})e^{\frac{t}{-\lambda_0}}dtdt = 
\frac{1}{\lambda_0}\int_0^t e^{t(\frac{1}{-\lambda_2}+\frac{1}{-\lambda_0})}(1-e^{\frac{t}{-\lambda_1}})dtdt\\
m_2 = \lambda_2(\lambda_1\frac{e^{-t(\frac{1}{\lambda_2}+\frac{1}{\lambda_1}+\frac{1}{\lambda_0})}-1}{\lambda_2\lambda_0+\lambda_2\lambda_1+\lambda_0\lambda_1}+\frac{1-e^{-t(\frac{1}{\lambda_2}+\frac{1}{\lambda_0})}}{\lambda_0+\lambda_2})
\end{split}
\end{equation}
Using the results for $dn_i$ from the properties of radioactive decay (\ref{eqn:propertiesRad})
\begin{equation}
\begin{split}
\frac{d\nu_{i+1}}{dt} = (1-P_{i})\frac{d\nu_{i}}{dt} =  (1-P_{i})(1-P_{i-1})\frac{d\nu_{i-1}}{dt}\\
\end{split}
\end{equation}
Which suggests the following formula 
\begin{equation}
\begin{split}
\frac{d\nu_{i+1}}{dt} = \frac{d\nu_{1}}{dt}\prod_{k=2}^{i} (1-P_{k}) = (1-P_{1})\frac{dP_0}{dt}\prod_{k=2}^{i} (1-e^{t\frac{1}{-\lambda_k}}) = -\frac{1}{\lambda_0}e^{\frac{t}{-\lambda_0}}\prod_{k=1}^{i} (1-e^{t\frac{1}{-\lambda_k}}) \\
\end{split}
\end{equation}
We will now prove the following using induction 
\begin{equation}\label{eqn:nucleiProduced}
\begin{split}
\frac{d\nu_{i}}{dt} = -\frac{1}{\lambda_0}e^{\frac{t}{-\lambda_0}}\prod_{k=1}^{i-1} (1-e^{t\frac{1}{-\lambda_k}})\forall i\in\mathbb{Z}, i>1\\
\end{split}
\end{equation}
Thus we have for the general case using the results for the daughter nuclei being produced (Equation \ref{eqn:nucleiProduced}) and the formula for the moles of the i-th generation nucleus (the top most equation in \ref{eqn:propertiesRad})
\begin{equation}
\begin{split}
m_i = \int_0^t P_i\frac{d\nu_i}{dt}dt = -\int_0^t e^{t\frac{1}{-\lambda_i}}\frac{1}{\lambda_0}e^{\frac{t}{-\lambda_0}}\prod_{k=1}^{i-1} (1-e^{t\frac{1}{-\lambda_k}})dt\\
m_i = -\frac{1}{\lambda_0}\int_0^te^{t(\frac{1}{-\lambda_0}+\frac{1}{-\lambda_1})}\prod_{k=1}^{i-1} (1-e^{t\frac{1}{-\lambda_k}})dt \\
\end{split}
\end{equation}
For the decay rate, we have
\begin{equation}
\begin{split}
r_i = \frac{d\nu_{i+1}}{dt} = -\frac{1}{\lambda_0}e^{\frac{t}{-\lambda_0}}\prod_{k=1}^{i} (1-e^{t\frac{1}{-\lambda_k}}) \\
r_i = \frac{d}{dt}\int_0^t 1-P_i\frac{d\nu_i}{dt}dt = -\frac{1}{\lambda_0}\frac{d}{dt}\int_0^t (1-e^{t\frac{1}{-\lambda_i}})e^{\frac{t}{-\lambda_0}}\prod_{k=1}^{i-1} (1-e^{t\frac{1}{-\lambda_k}})dt\\
r_i =  -\frac{1}{\lambda_0}e^{\frac{t}{-\lambda_0}}\prod_{k=1}^{i} (1-e^{t\frac{1}{-\lambda_k}})\\
\end{split}
\end{equation}






\end{document}
