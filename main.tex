\documentclass[12pt]{article}
\usepackage[utf8]{inputenc}
\usepackage{chemfig}
\usepackage[version=4]{mhchem}
\usepackage{amsfonts}
\usepackage{amsmath}
\usepackage{amssymb}
\usepackage{geometry}
\usepackage{mathabx}
\usepackage{relsize}
\usepackage{graphics}
\usepackage[colorlinks = true,
            linkcolor = blue,
            urlcolor  = blue,
            citecolor = blue,
            anchorcolor = blue]{hyperref}
%\usepackage{indentfirst}
\usepackage{tikz}
\usepackage{sidecap}
\usepackage{sidecap}
\geometry{
 a4paper,
 total={6.5in,0in},
 left= 15mm,
 top= 15mm,
 bottom=15mm,
 right = 15mm
 }
\title{Math for Energy}
\author{Marcos Perez}
\date{May 2022}

\begin{document}

\maketitle
\section{Daughter nucleus abundance in n-generation decay chain}
\subsection{Formula Proposition}
Where there 1 is one mole of the 0-th generation nucleus at $t=0$ there are $m_i$ moles of the $i-th$ generation nucleus at time $t$. We assume that the $i-th$ generation nucleus has a measurable e-folding time of $\lambda_i$. 
\begin{equation}
\begin{split}
m_0 = e^{\frac{t}{-\lambda_0}}\\
m_{i>0} = \frac{e^{ta}-1}{-a\lambda_0}\\
a = \sum_{k=0}^{i}{-\lambda_{i}^{-1}}
\end{split}
\end{equation}
\subsection{Formula Proof}
At a given time $t$ there are $m_i$ moles of the $i-th$ daughter nucleus. For $i>0$, there are $\nu_{i}$ daughter nuclei produced by the decay of the $i-1th$ generation in the decay chain, and this is the only source of any nuclei in the system. We assume there is no source of new 0-th generation nuclei. By the properties of radioactive decay
\begin{equation}\label{eqn:propertiesRad}
\begin{split}
m_i = \int_0^t P_id\nu_{i}\\
\nu_i = \int_0^t (1-P_{i-1})d\nu_{i-1}\\
\nu_0 = 0\quad \nu_1 = 1-P_0\\
\frac{d\nu_{i}}{dt} = (1-P_{i-1})\frac{d\nu_{i-1}}{dt}\forall i\in\mathbb{Z}, i>1\\
\end{split}
\end{equation}
Where the probability of a single nucleus with not decaying after a time $t$ is
\begin{equation}
\begin{split}
P_i = e^{-\frac{t}{\lambda_{i}}}\\
\end{split}
\end{equation}
Thus we make the substitutions to write the following recursive formula
\begin{equation}\label{eqn:recursiveForm}
\begin{split}
m_i = -\int_0^t P_i(1-P_{i-1})\frac{d\nu_{i-1}}{dt}dt\\
\end{split}
\end{equation}
Assuming there is no source of new 0-th generation nuclei
\begin{equation}
\begin{split}
m_1 = \int_0^te^{\frac{t}{-\lambda_1}}d\nu_0 = \int_0^te^{\frac{t}{-\lambda_1}}\frac{d}{dt}(1-P_0)dt = -\int_0^te^{\frac{t}{-\lambda_1}}\frac{dP_0}{dt}dt = \frac{1}{\lambda_0}\int_0^te^{\frac{t}{-\lambda_1}}e^{\frac{t}{-\lambda_0}}dt = \frac{\lambda_1}{\lambda_1+\lambda_0}(1-e^{-t(\frac{1}{\lambda_0}+\frac{1}{\lambda_1})}) \\
\end{split}
\end{equation}
For more insight on how to write the general case, we first compute using \ref{eqn:propertiesRad}
\begin{equation}
\begin{split}
m_2 = \int_0^t P_2d\nu_{2} = \int_0^t e^{\frac{t}{-\lambda_2}}d\nu_2\\
d\nu_{2} = (1-P_{1})\frac{d\nu_{1}}{dt}dt\\
m_2 = \int_0^t e^{\frac{t}{-\lambda_2}}(1-P_{1})\frac{d\nu_{1}}{dt}dt = -\int_0^t e^{\frac{t}{-\lambda_2}}(1-P_{1})\frac{dP_0}{dt}dtdt\\
\end{split}
\end{equation}
HAVE NOT CORRECTED BEYOND THIS POINT \hline In the general case (again only for $i>1$)
\begin{equation}
\begin{split}
m_{i+1} = -\int_0^t e^{-\frac{t}{\lambda_{i+1}}}(1-P_{i})\frac{d\nu_{i-1}}{dt}dt\\
m_{i+1} = \int_0^t e^{-\frac{t}{\lambda_{i+1}}}dt\\
\end{split}
\end{equation}
Simplifying the right side with the fundamental theorem of calculus
\begin{equation}
\begin{split}
m_{i+1} = \int_0^t e^{-\frac{t}{\lambda_{i+1}}} e^{-\frac{t}{\lambda_{i}}}\frac{dm_{i-1}}{dt}dt\\
\end{split}
\end{equation}
Now for $i=2$ using the previous results
\begin{equation}
\begin{split}
m_2 = \int_0^t e^{t\frac{1}{-\lambda_2}}d\nu_2 
 = \frac{1}{\lambda_0}\int_0^t e^{t\frac{1}{-\lambda_2}}e^{-\frac{t}{\lambda_{1}}}e^{\frac{t}{-\lambda_0}}dt =
 \frac{\lambda_1\lambda_2}{\lambda_0\lambda_1+\lambda_1\lambda_2+\lambda_0\lambda_2} e^{t\frac{1}{-\lambda_2}}e^{-\frac{t}{\lambda_{1}}}e^{\frac{t}{-\lambda_0}}\\
\end{split}
\end{equation}
Thus the formula holds for $i=0,1,\& \ 2$. Using induction, we will show that it holds for all generations of the decay chain. We will now show that if the formula holds for $i$, then it must hold for $i+1$. 
So we have
\begin{equation}
\begin{split}
m_i = \frac{\prod_{k=0}^ie^{\frac{t}{-\lambda}}}{\lambda_0\sum_{k=0}^i\frac{1}{\lambda_k}}\\
\end{split}
\end{equation}
By the recursive formula we derived (Eq. \ref{eqn:recursiveForm})
\begin{equation}
\begin{split}
m_{i+1} = -\int_0^t P_{i+1}\frac{dm_i}{dt} = -\int_0^t e^{\frac{t}{-\lambda_{i+1}}}\frac{d}{dt}\frac{\prod_{k=0}^ie^{\frac{t}{-\lambda}}}{\lambda_0\sum_{k=0}^i\frac{1}{\lambda_k}}dt = -\frac{1}{\lambda_0\sum_{k=0}^i\frac{1}{\lambda_k}}\int_0^t e^{\frac{t}{-\lambda_{i+1}}}\frac{d}{dt}\prod_{k=0}^ie^{\frac{t}{-\lambda}}dt\\
\end{split}
\end{equation}
Simplifying the right most side we have by first evaluating the derivative within the integral and using the properties of integration
\begin{equation}
\begin{split}
\frac{d}{dt}\prod_{k=0}^ie^{\frac{t}{-\lambda_k}} = \frac{d}{dt}e^{t\sum_{k=0}^i\frac{1}{-\lambda_k}} = e^{t\sum_{k=0}^i\frac{1}{-\lambda_k}}\sum_{k=0}^i\frac{1}{-\lambda_k}\\
m_{i+1} = -\frac{1}{\lambda_0\sum_{k=0}^i\frac{1}{\lambda_k}}\int_0^t e^{\frac{t}{-\lambda_{i+1}}}e^{t\sum_{k=0}^i\frac{1}{-\lambda_k}}\sum_{k=0}^i\frac{1}{-\lambda_k}dt = -\frac{\sum_{k=0}^i\frac{1}{-\lambda_k}}{\lambda_0\sum_{k=0}^i\frac{1}{\lambda_k}}\int_0^t e^{\frac{t}{-\lambda_{i+1}}}e^{t\sum_{k=0}^i\frac{1}{-\lambda_k}}dt
\end{split}
\end{equation}
Simplifying the constant in front of the integral and evaluating the integral yields 
\begin{equation}
\begin{split}
-\frac{\sum_{k=0}^i\frac{1}{-\lambda_k}}{\lambda_0\sum_{k=0}^i\frac{1}{\lambda_k}} = \frac{1}{-\lambda_0}\\
\int_0^t e^{\frac{t}{-\lambda_{i+1}}}e^{t\sum_{k=0}^i\frac{1}{-\lambda_k}}dt = \int_0^t e^{t\sum_{k=0}^{i+1}\frac{1}{-\lambda_k}}dt = (e^{t\sum_{k=0}^{i+1}\frac{1}{-\lambda_k}}-1)dt  
\frac{1}{-\lambda}\int_0^t e^{\frac{t}{-\lambda_{i+1}}}e^{t\sum_{k=0}^i\frac{1}{-\lambda_k}}dt =  \\
m_{i+1} = \frac{1}{-\lambda}\int_0^t e^{\frac{t}{-\lambda_{i+1}}}e^{t\sum_{k=0}^i\frac{1}{-\lambda_k}}dt = \frac{e^{t\sum_{k=0}^i\frac{1}{-\lambda_k}}}{-\lambda\sum_{k=0}^i\frac{1}{-\lambda_k}}\\
\blacksquare
\end{split}
\end{equation}









\end{document}
