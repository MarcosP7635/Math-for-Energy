\documentclass[12pt]{article}
\usepackage[utf8]{inputenc}
\usepackage{chemfig}
\usepackage[version=4]{mhchem}
\usepackage{amsfonts}
\usepackage{amsmath}
\usepackage{amssymb}
\usepackage{geometry}
\usepackage{mathabx}
\usepackage{relsize}
\usepackage{graphics}
\usepackage[colorlinks = true,
            linkcolor = blue,
            urlcolor  = blue,
            citecolor = blue,
            anchorcolor = blue]{hyperref}
%\usepackage{indentfirst}
\usepackage{tikz}
\usepackage{sidecap}
\usepackage{sidecap}
\geometry{
 a4paper,
 total={6.5in,0in},
 left= 15mm,
 top= 15mm,
 bottom=15mm,
 right = 15mm
 }
\title{Methods to Increase Power Density}
\author{Marcos Perez}
\date{May 2022}

\begin{document}
\maketitle
Consider a system of $n$ capacitors (or any electric storage system with high power density) all completely charged by a source with an energy density of $10^8$Joules/gram but a power density of $\rho_p$. The capacitors each discharge a power $P$ in watts (for our purposes assume the capacitors discharge at their maximum rate until they are completely discharged).\\
If a car consumes $10^5.5$W, then it would consume $10^{-2.5}$gram/second. Thus 1kg would last \\
If the capacitors have an energy density$ \rho_E,$ and n of them at any given time are sufficient to power the car, and there are N capacitors connected to the battery, which has a power density $\rho_P$. 
If n capacitors are fully charged, they can sufficiently power the car for t seconds.
What n and N are necessary for the car to drive continuously until $\rho_P$ is insufficient to charge n capacitors?
$t * N / n $is the number of seconds the battery has to charge n capacitors. It takes the battery $\rho_E * n / \rho_P$ seconds to do this. Thus we have
\begin{equation}
\begin{split}
    \frac{tN}{n} \le \frac{\rho_Em_Cn}{m_B \rho_P} \\
    m_B\rho_P \le \frac{n^2\rho_Em_C}{tN}\\
    (10^5)(10^{-1}) \le \frac{(10^{4.5}).3}{2(10^{1.5})}  \\
\end{split}
\end{equation}
Tesla car batteries have a specific power of 1.5W/g
(\href{
https://www.tesla.com/pt_PT/blog/bit-about-batteries
}{
tesla blog
},
\href{
https://en.m.wikipedia.org/wiki/Tesla_Model_S
}{
Tesla model S wiki
}) and supercapacitors have specific powers of 15 W/g \href{
https://en.m.wikipedia.org/wiki/Supercapacitor
}{
https://en.m.wikipedia.org/wiki/Supercapacitor
}. Thus we would need $10^{4.5}$ g of supercapacitors at a time 
powering the car. We can store on the order of $10^5$g in the car, 
so $N=2n$ and a specific energy of 0.3 J/g. We have for $t = 10^{4.5+.5-5.5} = 10^{-.5}$. Since the third line isn't true, this
system would not work. \\
So the most mass we could fit is $10^5$g. A fraction f of the 100kg is powering the car at a time. This allows the secondary battery to power the car for $m $
For a power consumption $P$ we have $t=\rho_{CE}/(\rho_{CP}P) = \rho_{CE}/\rho_{CP}$. 
What if we used lithium ion instead? In this case $t = 10^{} = $. 


\end{document}